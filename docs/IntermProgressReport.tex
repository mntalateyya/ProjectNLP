\documentclass{article}
\usepackage[a4paper, margin=2.5cm]{geometry}
\usepackage{amsmath, amssymb}
\usepackage{fontspec}
\usepackage{unicode-math}
\usepackage[dvipsnames]{xcolor}
\setmainfont[Ligatures=TeX]{TeX Gyre Pagella}
\setmathfont[Ligatures=TeX]{TeX Gyre Pagella Math}

\title{Interim Progress Report}
\author{Haya Al-Kuwari, Akhyar Kamili, Mohammed Nurul Hoque, Abubaker Omar}
\date{\today}

\begin{document}
    \maketitle
    \noindent\rule{\textwidth}{1pt}
    This report presents our group's progress in the project and our implementation plan so far.

    \section{Initial Experiments}
    We experimented with Stanford CoreNLP library using simple sentences. We found the idea
    \texttt{tokregex \& semgrex} particularly appealing because it simplifies a lot of processing as
    regex matching-like operations. Unfortunately, the CoreNLP server takes up too much space and
    time when these features are used, while it is quite snappy when used just for regular
    operations like sentence-splitting, tagging and parsing.

    \section{Implementation Plan}
    We decided to implement a subset of the \texttt{semgrex} module that is sufficient for our purpose. ...

    \section{Results}
    We implemented a prototype for a splitting sentences and matching a template agains each
    sentence. We tested it with an actual template and document. Figure <insert ref> shows some of
    the results.
    
\end{document}
