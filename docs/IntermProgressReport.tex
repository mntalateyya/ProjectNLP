\documentclass{article}
\usepackage[a4paper, margin=2.5cm]{geometry}
\usepackage{amsmath, amssymb}
\usepackage{fontspec}
\usepackage{unicode-math}
\usepackage[dvipsnames]{xcolor}
\setmainfont[Ligatures=TeX]{TeX Gyre Pagella}
\setmathfont[Ligatures=TeX]{TeX Gyre Pagella Math}

\title{Interim Progress Report}
\author{Haya Al-Kuwari, Akhyar Kamili, Mohammed Nurul Hoque, Abubaker Omer}
\date{\today}

\begin{document}
    \maketitle
    \noindent\rule{\textwidth}{1pt}
    This report presents our group's progress in the project and our implementation plan so far.

    \section{Initial Experiments}
    We experimented with Stanford CoreNLP library using simple sentences. We found the idea of using
    \texttt{tokregex \& semgrex} to generate and answer questions particularly appealing because it simplifies a lot of processing as
    regex matching-like operations. Unfortunately, the CoreNLP server takes up huge amount of space and
    time when these features are used, while it is quite fast when used just for regular
    operations like sentence-splitting, tagging and parsing.

    \section{Implementation Plan}
    We implemented two modules to deal with question generation. Firstly, we implemented the 'Document' class, to simplify working with text documents in which we only need to supply its method \texttt{generateQuestionsFromPattrens} with a list of tuples of pattren and function for createing questions from these matchings. We, also, implemented a class of dependency graph and POS tags for a sentences, a subset of semgrex increase performance. Those strucutres are supposed to help us simplfy and speed up the performance based on the task.

    \section{Results}
    We implemented a prototype for a splitting sentences and matching a template agains each
    sentence. We tested it with an actual template and document. Figure <insert ref> shows some of
    the results.
    
\end{document}
