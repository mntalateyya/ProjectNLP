\documentclass{article}
\usepackage[a4paper, margin=2.5cm]{geometry}
\usepackage{amsmath, amssymb}
\usepackage{fontspec}
\usepackage{unicode-math}
\usepackage[dvipsnames]{xcolor}
\usepackage{listings}
\setmainfont[Ligatures=TeX]{TeX Gyre Pagella}
\setmathfont[Ligatures=TeX]{TeX Gyre Pagella Math}

\newcommand{\code}[1] {
	\noindent\rule{\textwidth}{0.4pt}
	#1
	\vspace{-0.3em}
	\par\noindent\rule{\textwidth}{0.4pt}
	\vspace{-1.7\baselineskip}
	\lstinputlisting[language=Python]{#1}
}

\definecolor{codegreen}{RGB}{133, 153, 00}
\definecolor{commentgray}{RGB}{147, 161, 161}
\definecolor{codegray}{RGB}{88, 110, 117}
\definecolor{codecyan}{RGB}{42, 161, 152}
\definecolor{backcolour}{RGB}{253, 246, 227}

\lstdefinestyle{mystyle}{
    backgroundcolor=\color{backcolour},
    commentstyle=\color{commentgray}\slshape,
    keywordstyle=\color{codegreen},
    numberstyle=\tiny\color{codecyan},
    stringstyle=\color{codecyan},
	basicstyle=\ttfamily\color{codegray}\footnotesize,
    numbers=left,
    showstringspaces=false,
}

\lstset{style=mystyle}

\title{Interim Progress Report}
\author{Haya Al-Kuwari, Akhyar Kamili, Mohammed Nurul Hoque, Abubaker Omer}
\date{\today}

\begin{document}
    \maketitle
    \noindent\rule{\textwidth}{1pt}
    This report presents our group's progress in the project and our implementation plan so far. We
    experimented with Stanford CoreNLP and decided to use its tokenizer, tagger and dependency
    parser. We also decided to implement parts of stanford's \texttt{semgrex} ourselves. Finally we
    modified and simplified our pipeline as described below.

    \section{Initial Experiments}
    We experimented with Stanford CoreNLP library using simple sentences. We found the idea of using
    \texttt{tokregex \& semgrex} to generate and answer questions particularly appealing because it
    simplifies a lot of processing as regex matching-like operations. Unfortunately, the CoreNLP
    server takes up huge amount of space and time when these features are used, while it is quite
    fast when used just for regular operations like tokenizing, tagging and parsing.

    \section{Tested Prototypes}
    We implemented two modules to deal with question generation. Firstly, we implemented the
    'Document' class, to simplify working with text documents in which we only need to supply its
    method \texttt{generateQuestionsFromPattrens} with a list of tuples of pattren and function for
    createing questions from these matchings. We, also, implemented a class \texttt{DepGraph} which
    is a subset of the functionality of stanford's \texttt{semgrex}. It takes relations of the form
    ``node $x$ has tag $VBD$ and is a governor of a \textit{nsubj} relation to a node $y$ that has
    tag \textit{NNP} and also a \textit{dobj} relation to a node $z$ of unspecified tag'' (In
    CoreNLP notation \texttt{\{tag:VBD\} >nsubj \{tag: NNP\} >dobj \{\}}). It returns indices of the
    words matching the nodes in the same order. The current implementation matches only in a single
    level but we will expand it to match recursively on the parse tree.

    \section{Results}
    We tested out implementation with an actual template and documents from the data set. We managed to retrieve questions of the form "What Hazanavicius fantasized about? " from sentence like "Hazanavicius fantasized about making a silent film". Figure "/src/test.py" shows some of
    the results.

    \begin{figure}
        \code{../src/test.py}
    \end{figure}
    
\end{document}
