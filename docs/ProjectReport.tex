\documentclass{article}
\usepackage[a4paper, margin=2.5cm]{geometry}
\usepackage{amsmath, amssymb}
\usepackage{fontspec}
\usepackage{unicode-math}
\usepackage[dvipsnames]{xcolor}
\usepackage{listings}

\title{A system for question and answer generations}
\author{Haya Al-Kuwari, Akhyar Kamili, Mohammed Nurul Hoque, Abubaker Omar}
\date{\today}

\begin{document}
    \maketitle
    \begin{abstract}
        This paper documents our NLP project, a question generation and answering system from
        article documents. Our system generates question using tree manipulation of the dependency
        parse of sentences and simple ranking. Answer generation is done using maximum lexical
        overlap.
    \end{abstract}        
    \section{Introduction}
    \section{Usage}
        \subsection{Generating questions from a document}
        You can generate a question using a document using command line: 
        \begin{verbatim}
        $ ./ask document.txt n 
        \end{verbatim}
        where \texttt{document.txt} is the document file that you want to generate questions from, 
        and \texttt{n} is the number of questions you want to generate.
        \subsection{Answering document-related questions from a user}
        You can get answers that is related to the document you will be providing using command line: 
        \begin{verbatim}
        $ ./answer document.txt questions.txt 
        \end{verbatim}
        where \texttt{document.txt} is the document that you are giving questions about, 
        and {questions.txt} is the questions you are giving in a text file. 
    \section{Implementation}
    \section{Results}
\end{document}
